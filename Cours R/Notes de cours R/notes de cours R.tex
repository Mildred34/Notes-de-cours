\documentclass[12pt,a4paper]{article}
\usepackage[utf8]{inputenc}
\usepackage{graphicx}
\graphicspath{{../Images/}}
\usepackage{amsmath}
\usepackage{amsfonts}
\usepackage{amssymb}
\usepackage{hyperref}
\usepackage[margin=1in]{geometry}
\usepackage{subfig}
\usepackage{float}

\author{Thibaut Marmey}

\title{Notes de cours R}
\begin{document}
	\maketitle

\begin{normalsize}
\tableofcontents
\end{normalsize}

\section{Programmation R}
\subsection{Généralité}
\begin{itemize}
\item \href{https://linuxconfig.org/rstudio-on-ubuntu-18-04-bionic-beaver-linux}{\textit{lien internet : }doc d'installation de R et Rstudio}
\item Utilisation de la documentation "aide" : \textit{help()} ou \textit{help("nom de la fonction")} ou  \textit{?log}
\item Documentation plus détaillée sur internet : \textit{help.start()}
\end{itemize}

\subsection{Fonctions}
\begin{itemize}
\item Tester le type d'une variable :
\begin{itemize}
\item \textit{is.character(var)}
\item \textit{is.numeric(var)}
\item \textit{is.logical(var)}
\end{itemize}
\item Spécifier le typage de la variable :
\begin{itemize}
\item \textit{as.numeric(var)}
\item \textit{as.logical(var)}
\item \textit{as.character(var)}
\end{itemize}
\item Renvoyer l'entier inférieur : \textit{floor(nb)}
\item Renvoyer l'entier supérieur : \textit{ceiling(nb)}
\item Arrondir à l'entier le plus proche : \textit{round(nb)}
\item fonction trigo : \textit{cos(angleRad), sin, ...}
\end{itemize}

\end{document}